The goal of ensemble methods is to combine the predictions of several models built with a given learning algorithm in order to improve generalizability / robustness over a single model.

Ensembles are well known for their effect of increasing overall accuracy and
overcoming over-fitting, as well as allowing parallelism. They have successfully
been used in various multi-label problems.

\subsubsection{Ensemble theory}

(WIKI) An ensemble is itself a supervised learning algorithm, because it can be trained and then used to make predictions. The trained ensemble, therefore, represents a single hypothesis. This hypothesis, however, is not necessarily contained within the hypothesis space of the models from which it is built. Thus, ensembles can be shown to have more flexibility in the functions they can represent. This flexibility can, in theory, enable them to over-fit the training data more than a single model would, but in practice, some ensemble techniques (especially bagging) tend to reduce problems related to over-fitting of the training data.

\paragraph{Categories}

(Scikit-learn website) Two families of ensemble methods are usually distinguished:
In averaging methods, the driving principle is to build several models independently and then to average their predictions. On average, the combined model is usually better than any of the single model because its variance is reduced. Examples include Bagging methods, Forests of randomized trees...

By contrast, in boosting methods, models are built sequentially and one tries to reduce the bias of the combined model. The motivation is to combine several weak models to produce a powerful ensemble. Examples include AdaBoost, Gradient Tree Boosting, ...


\paragraph{Diversity}

(WIKI) Empirically, ensembles tend to yield better results when there is a significant diversity among the models. Many ensemble methods, therefore, seek to promote diversity among the models they combine. Although perhaps non-intuitive, more random algorithms (like random decision trees) can be used to produce a stronger ensemble than very deliberate algorithms (like entropy-reducing decision trees).[8] Using a variety of strong learning algorithms, however, has been shown to be more effective than using techniques that attempt to dumb-down the models in order to promote diversity.


\cite{kuncheva2003measures} Diversity among the members of a team of classifiers is deemed to be a key issue in classifier
combination. However, measuring diversity is not straightforward because there is no generally accepted formal
definition. We have found and studied ten statistics which can measure diversity among binary classifier outputs
(correct or incorrect vote for the class label): four averaged pairwise measures (the Q statistic, the correlation,
the disagreement and the double fault) and six non-pairwise measures (the entropy of the votes, the difficulty
index, the Kohavi-Wolpert variance, the interrater agreement, the generalized diversity, and the coincident failure
diversity). Four experiments have been designed to examine the relationship between the accuracy of the team and
the measures of diversity, and among the measures themselves. Although there are proven connections between
diversity and accuracy in some special cases, our results raise some doubts about the usefulness of diversity
measures in building classifier ensembles in real-life pattern recognition problems.


\subsubsection{Existing Methods in ML}

Methods in Scikit-learn package: Forests of randomized trees; AdaBoost; Gradient Tree Boosting  (\url{http://scikit-learn.org/stable/modules/ensemble.html})

\paragraph{Bootstrap aggregating (bagging)}
Bootstrap aggregating, often abbreviated as bagging \cite{breiman1996bagging}, involves having each model in the ensemble vote with equal weight.

\paragraph{Random forest}

Random forest \cite{breiman2001random}: operate by constructing a multitude of decision trees at training time and outputting the class that is the mode of the classes output by individual trees. 

\paragraph{AdaBoost}

AdaBoost \cite{freund1997decision}, short for "Adaptive Boosting", in the sense that subsequent classifiers built are tweaked in favor of those instances misclassified by previous classifiers. 

\paragraph{Gradient boosting}
\label{sec:ml:GradientBoosting}
Gradient boosting \cite{friedman2001greedy} is a machine learning technique for regression problems, which produces a prediction model in the form of an ensemble of weak prediction models, typically decision trees. It builds the model in a stage-wise fashion like other boosting methods do, and it generalizes them by allowing optimization of an arbitrary differentiable loss function. The gradient boosting method can also be used for classification problems by reducing them to regression with a suitable loss function.

\paragraph{Bayesian model averaging}

Bayesian model averaging (BMA) \cite{hoeting1999bayesian}  is a pratical ensemble technique that seeks to approximate the Bayes Optimal Classifier by sampling hypotheses from the hypothesis space, and combining them using Bayes' law.

\paragraph{Ensemble of Classifier Chains}

Classifier chains is a machine learning method for problem transformation in multi-label classification. It combines computational efficiency of Binary Relevance method and possibility to use dependencies between labels for classification.  

\cite{read2011classifier} In Ensemble of Classifier Chains (ECC) several CC classifiers can be trained with random order of chains (i.e. random order of labels) on a random subset of data set. Labels of a new instance are predicted by each classifier separately. After that, the total number of predictions or "votes" is counted for each label. The label is accepted if it was predicted by a percentage of classifiers that is bigger than some threshold value.

ECC achieves better performance than binary relevance because it can exploit correlations in the label sets.\cite{mlsp2}

\subsubsection{Applications on Songbirds}

\paragraph{Ensemble logistic regression and gradiant bootsting classfiers \cite{Massaron13}}

After estimating the probabilities of species being present in a time unit in a
sound file (as for as logistic regression models using its link function, as for as hinge
regression models by rescaling and clipping the results), simply averaged logistic and hinge
probability results and therefore obtained a first ensemble forecast of the presence of every
singular species in every row of every target transposed test MFCC matrix.

In order to turn the results relative to single time units into overall probabilities of species
presence in each sound file, the author empirically experimented that using for each test
matrix a moving average of 200 rows and retaining for each species the maximum
probability result allowed to obtain a prediction whose public AUC was 0.87791 and its
private one was 0.87120.

Noticing, by direct inspection of the fitted results on the train set and on the test results, that
the estimations had surely an high recall of the species (systematically a large number of
species had high scores for each test MFCC matrix, pointing out the likelihood of many false
positives) but were likely lacking the necessary precision to reach higher scores on the
Kaggle’s leaderboard, the author decided to integrate the linear models by a different
approach based on gradient boosting classifiers (Section \ref{sec:ml:GradientBoosting}), as implemented in the Scikit-learn
library in Python \cite{pedregosa2011scikit} (using the function {\em GradientBoostingClassifier}).

The underlying idea was that gradient boosting classifier (GBC), allowing interactions, has
surely less bias than the linear models (thus an increased precision) but were suffering from
an higher variance in estimates.

The ensemble approach required to create a new training dataset, resampling the initial one,
in order to obtain, for each target species, all the examples of the target species itself and a
5\% of examples available in each training MFCC matrix.

By examining some randomly chosen predictions from the test set, it can be observed that, as
depicted in figure 1 for test sound file no. 500, an ensemble of logistic and hinge
regressions tends to polarize the results in high and low probability ends and to mark many
species as possibly present in the sound file.

The same graphical inspection for the GBC model reveals a completely different patter,
being the estimated probabilities limited in value, with spikes relative to only the most likely
ones. Such a pattern, repeated all over the test file, confirms the author’s expectation of the
gradient boosting approach to point out only the species certainly present with an high
probability and confidence, thus penalizing the recall of other species whose presence is
suspected, but with equivalent certainty, confirmed.

Finally, on the basis of such an insight, it has therefore been created an ensemble bringing
together the results from both the averaged linear models and the gradient boosting
classifier, by means of an harmonic mean, brought the final result of a public AUC of
0.89575 and a private one of 0.89041.

It is observed in figure 3 how the previously polarized predictions have naturally arranged
themselves into probability tiers, allowing a better probability estimation, as for as the AUC
measure.

The proposed approach highlights how an ensemble mixing high bias / low variance models
and low bias / high variance ones may prove an effective strategy in bioacoustics problems.
Moreover, the gradient boosting classifiers are a tree based machine learning methodology
that should better explored. There are
furthermore open opportunities in further tuning of models’ hyper-parameters and in
simplifying the ensemble strategy.

\paragraph{Linear blending \cite{chennovel13}}

Identify prominent features from windowing mfccs with overlap and leverage them to build an ensemble classifier which is a blend of different classifiers (Gradient Boosting Tree
models, Random Forest models and Lasso and elastic-net regularized generalized linear model etc). 

As blending algorithm we use a simple basic blending - linear blending. A linear blender is easy to
implement. The most basic blending method is to compute the final prediction simply as the mean
over all the predictions in the ensemble. Better results can be obtained, if the final prediction is
given by a linear combination of the ensemble predictions. In this case, the combination coefficients
have to be determined by optimization procedure, in general by regularized linear regression. In our
case, All inputs (the predictions) are normalized to [0...+1]. We use linear blending approach and
the coefficients are determined by cross-validation performance on average precision. For validation
we use 10-fold cross-validation.

\paragraph{Ensemble of Classifier Chain swith Random Forest}

\cite{mlsp2} implement ECC \cite{read2011classifier} with a Random Forest (RF) as
the base-SISL classifier, hence we call the proposed
classifier ECC-RF. Because RF outputs a probability,
the ensemble can be viewed as an instance of the En-
semble of Probabilistic Classifier Chains (EPCC) al-
gorithm. Therefore it is
reasonable to aggregate probabilities from each SISL
classifier rather than 0/1 votes. The aggregated prob-
abilities are used as the score-functions for each class.

\paragraph{Ensemble of randomized decision trees}

\cite{Lasseck13} For each sound class an ensemble of
randomized decision trees (sklearn.ensemble.ExtraTreesRegressor) is applied. The number
of estimators is chosen to be twice the number of selected features per class but not greater
than 500. The winning solution considers 4 features when looking for the best split and
requires a minimum of 3 samples to split an internal node. During 12 -fold cross validation
the probability of each sound class in all test files is predicted and at the end, after removing
the lowest and highest value, averaged.

By ranking feature importance returned from the decision trees during training one can find
important segments to identify each sound class. 




\subsubsection{Other Thoughts}

\paragraph{unknown class}

In practice, the rejection option selects the objects that cannot be assigned to any class with sufficient confidence.

