  The features used can have a large impact on classification results. For this project, we've chosen to implement a
large set of features that have been used in previous successful attempts at birdsong classification. We will ultimately
choose a smaller subset of these features that provide the most discriminatory power. 
  Our current prototype utilizes Mel frequency Cepstral Coefficient (MFCC) features. The MFCC features are common in
speech recognition, but can be sensitive to the presence of noise [1]; for this reason, we expect that they may be
useful features for classifying the clear lab data but less useful for the noisy outdoor data. 
  The other features that we intend to explore can be classified broadly into two categories: spectrum-wide features and
segment features. The spectrum-wide features describe qualities of the full spectrograph for a given recording, while
the segment features describe qualities of individual segmented portions of the spectrograph.
  For spectrum-wide features, we have already explored using 13-dimensional MFCC vector using Wojcicki's MATLAB
implementation [2]. In addition, we plan on using the min, max, average, and standard deviation of frequency values over
the entire spectrograph, and on dividing the spectrograph into N equally spaced frequency bands and taking the same
statistics for each frequency band as features. Lasseck had success with this method in 2013 using N=16 [3].
  For segment features, we will use the same basic statistics on a per-segment basis. In addition, we will explore the
usefulness of many of the other segment features that were used successfully by Briggs et al. in 2012. These can be divided 
into two different categories - "mask descriptors", which describe the general shape of a segment, and "profile statistics", 
which describe the frequency and time profiles within segments. The mask descriptors include bandwidth,duration, area, 
perimeter, non-compactness, and rectangularity; the profile statistics include profile uniformity, mean, variance,
skewness, and kurtosis. The final set of segment features that we will explore are the Histogram of Gradients (HOG), 
which have successfully been used for songbird classification by both Briggs in 2012 [4] and Fodor in 2013 [5]. 


[1] Tyagi, V.; Wellekens, Christian, "On desensitizing the Mel-Cepstrum to spurious spectral components for Robust
Speech Recognition," Acoustics, Speech, and Signal Processing, 2005. Proceedings. (ICASSP '05). IEEE International
Conference on , vol.1, no., pp.529,532, March 18-23, 2005

[2] http://www.mathworks.com/matlabcentral/fileexchange/32849-htk-mfcc-matlab/content/mfcc/mfcc.m

[3] Lasseck, Mario. "Bird Song Classification in Field Recordings: Winning Solution for NIPS4B 2013 Competition."

[4] F. Briggs, B. Lakshminarayanan, L. Neal, X. Z. Fern,
R. Raich, S. J. K. Hadley, A. S. Hadley, and M. G.
Betts. Acoustic classification of multiple simultane-
ous bird species: A multi-instance multi-label approach.
In The Journal of the Acoustical Society of America,
131:4640, 2012.

[5] Fodor G. (2013) The Ninth Annual MLSP Competition: First place. Machine Learning for Signal
Processing (MLSP), 2013 IEEE International Workshop on, Digital Object Identifier:
10.1109/MLSP.2013.6661932 Publication Year: 2013, Page(s): 1- 2

