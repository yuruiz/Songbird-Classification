\cite{BriggsMLSP13} It is important to gain a better understanding of bird behavior
and population trends. Birds respond quickly to environmen-
tal change, and may also tell us about other organisms (e.g.,
insects they feed on), while being easier to detect. Traditional
methods for collecting data about birds involves costly hu-
man effort. A promising alternative is acoustic monitoring.
There are many advantages to recording audio of birds com-
pared to human surveys, including increased temporal and
spatial resolution and extent, applicability in remote sites, re-
duced observer bias, and potentially lower cost. However, it is
an open problem for signal processing and machine learning
to reliably identify bird sounds in real-world audio data col-
lected in an acoustic monitoring scenario. Some of the major
challenges include multiple simultaneously vocalizing birds,
other sources of non-bird sound (e.g., buzzing insects), and
background noise like wind, rain, and motor vehicles.
